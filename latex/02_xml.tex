\chapter{Namespace XML}

Les documents XML bien formés répondront à l'arborescence présentée ci-dessous.

Les hypothèse suivantes sont également formulées :
\begin{itemize}
    \item Les documents XML à parser ne contiendront pas de DTD interne
    \item Les Process Instruction (PI) ne contiendront que des attributs
    \item Aucune référence ne sera présente dans les documents XML\\
\end{itemize}


\section{Diagramme de classes}
    Les classes ici décrites seront rattachées au \lstinline$namespace Xml$ défini en C++.

    \begin{landscape}
    \begin{figure}[h!]
        \centering
        \includegraphics[width=0.9\linewidth]{images/xml-uml.pdf}
        \caption{Diagramme de classes de l'arborescence XML}
        \label{classDiagram}
    \end{figure}
    \end{landscape}


\section{Classes}
    \subsection{Object}
        Pour optimiser et eviter au maximum la redondance de donner dans notre arbre d'h\'eritage, nous nous somme inspirer de la tres c\'elebre librarie \textit{Qt} avec sont \lstinline$QObject$. Ainsi nous avons notre \lstinline$Xml::Object$ offrant les avantages que vous trouverez dans les sous sections suivantes.

    \subsection{Node}
        Ainsi, nous partons du principe qu'un document XML n'est en faite qu'un arbre aillant des noeux (\lstinline$Node$) \'etant des objects XML, mais de natures differents (commentaires, text, element ...). Certains de ces noeux serons des feuilles de l'arbre pour le noeux de commentaire par exemple.

        C'est la d\'ej\`a un avantage du \lstinline$Xml::Object$ car chaque \lstinline$Node$ connais sont \lstinline$Object$ parent \'a l'aide de \lstinline$mParent$ pouvant etre un Element ou un Document. Ainsi, on peut \'a partir d'un noeud, retrouver le \lstinline$Document$ dans lequel il se trouve simplement en remontant l'arbre.

    \subsection{DocumentNode}
        Le \lstinline$DocumentNode$ est un \lstinline$Node$, mais aillant une particulariter semantique~: seul ses classes fillent peuvent avoir pour parent, un \lstinline$Element$ par hertiage de \lstinline$Node$, mais aussi un \lstinline$Document$ au contraire de la classe \lstinline$Text$ ne pouvant avoir pour parent qu'un \lstinline$Element$.

    \subsection{Document}
        Un \lstinline$Document$ est un \lstinline$Node$ aillant une composition de \lstinline$DocumentNode$. Parmis ces \lstinline$DocumentNode$, un seul et unique \lstinline$Element$ racine compose cette liste. Mais afin de pouvoir retrouver cette racine du document en complexit\'ee $O(1)$, \lstinline$DocumentNode$ dispose aussi d'un attributs \lstinline$mRoot$ dedi\'e \`a cette tache.

        Cette m\^eme liste de \lstinline$DocumentNode$ est ordonn\'ee pour pouvoir garentir l'ordre des noeux au chargement et \`a l'export du document \textit{XML}. L'\lstinline$Element$ racine compose cette liste pour eviter de gerer deux listes de noeux (ceux avant et ceux apr\`es).

    \subsection{Comment}
        Contient la description d'un commentaire sous forme de chaîne de caractères.

    \subsection{ProcessingInstruction}
        En accord avec l'hypothèse énoncée précédemment, une PI ne contient qu'un nom et une liste ordonnée d'attributs

    \subsection{Text}
        Un noeud texte est naturellement décrit par une chaîne de caractères. Celui-ci ne dérive pas de la classe MiscNode puisque, si son contenu s'apparente à celui d'un noeud Comment, il ne peut en revanche être contenu par une instance de Document. Il est alors nécessaire d'ajouter un niveau de spécialisation afin d'éviter une relation incohérente.

    \subsection{Element}
        Un élément XML dispose d'un nom de balise, d'un ensemble ordonné d'attributs et d'un ensemble ordonné de noeuds fils. Cet élément peut être une balise vide ($<br/>$).
